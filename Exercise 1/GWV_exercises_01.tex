%%%%%%%%%%%%%%%%%%%%%%%%%%%%%%%%%%%%%%%%%%
% Short Sectioned Assignment LaTeX Template Version 1.0 (5/5/12)
% This template has been downloaded from: http://www.LaTeXTemplates.com
% Original author:  Frits Wenneker (http://www.howtotex.com)
% License: CC BY-NC-SA 3.0 (http://creativecommons.org/licenses/by-nc-sa/3.0/)
%%%%%%%%%%%%%%%%%%%%%%%%%%%%%%%%%%%%%%%%%

%----------------------------------------------------------------------------------------
%	PACKAGES AND OTHER DOCUMENT CONFIGURATIONS
%----------------------------------------------------------------------------------------

\documentclass[paper=a4, fontsize=11pt]{scrartcl} % A4 paper and 11pt font size

% ---- Entrada y salida de texto -----

\usepackage[T1]{fontenc} % Use 8-bit encoding that has 256 glyphs
\usepackage[utf8]{inputenc}
%\usepackage{fourier} % Use the Adobe Utopia font for the document - comment this line to return to the LaTeX default

% ---- Idioma --------

\usepackage[spanish, es-tabla]{babel} % Selecciona el español para palabras introducidas automáticamente, p.ej. "septiembre" en la fecha y especifica que se use la palabra Tabla en vez de Cuadro

% ---- Otros paquetes ----

\usepackage{url} % ,href} %para incluir URLs e hipervínculos dentro del texto (aunque hay que instalar href)
\usepackage{amsmath,amsfonts,amsthm} % Math packages
%\usepackage{graphics,graphicx, floatrow} %para incluir imágenes y notas en las imágenes
\usepackage{graphics,graphicx, float} %para incluir imágenes y colocarlas

% Para hacer tablas comlejas
%\usepackage{multirow}
%\usepackage{threeparttable}

%\usepackage{sectsty} % Allows customizing section commands
%\allsectionsfont{\centering \normalfont\scshape} % Make all sections centered, the default font and small caps

\usepackage{fancyhdr} % Custom headers and footers
\pagestyle{fancyplain} % Makes all pages in the document conform to the custom headers and footers
\fancyhead{} % No page header - if you want one, create it in the same way as the footers below
\fancyfoot[L]{} % Empty left footer
\fancyfoot[C]{} % Empty center footer
\fancyfoot[R]{\thepage} % Page numbering for right footer
\renewcommand{\headrulewidth}{0pt} % Remove header underlines
\renewcommand{\footrulewidth}{0pt} % Remove footer underlines
\setlength{\headheight}{13.6pt} % Customize the height of the header

\numberwithin{equation}{section} % Number equations within sections (i.e. 1.1, 1.2, 2.1, 2.2 instead of 1, 2, 3, 4)
\numberwithin{figure}{section} % Number figures within sections (i.e. 1.1, 1.2, 2.1, 2.2 instead of 1, 2, 3, 4)
\numberwithin{table}{section} % Number tables within sections (i.e. 1.1, 1.2, 2.1, 2.2 instead of 1, 2, 3, 4)

\setlength\parindent{0pt} % Removes all indentation from paragraphs - comment this line for an assignment with lots of text

\newcommand{\horrule}[1]{\rule{\linewidth}{#1}} % Create horizontal rule command with 1 argument of height






%----------------------------------------------------------------------------------------
%	TÍTULO Y DATOS DEL ALUMNO
%----------------------------------------------------------------------------------------

\title{	
\normalfont \normalsize 
\textsc{\textbf{Grundlagen der Wissensverarbeitung} \\ Computer Science \\ Universität Hamburg} \\ [25pt] % Your university, school and/or department name(s)
~\\
~\\
~\\
\horrule{0.5pt} \\[0.4cm] % Thin top horizontal rule
\Huge Übung 1: Search Space Design \\ % The assignment title
\horrule{2pt} \\[0.5cm] % Thick bottom horizontal rule
~\\
~\\
}

\author{Rafael Ruz Gómez\\Miguel Robles Urquiza} % Nombre y apellidos

\date{\normalsize 5 November 2017} % Incluye la fecha actual

%----------------------------------------------------------------------------------------
% DOCUMENTO
%----------------------------------------------------------------------------------------

\begin{document}

\maketitle % Muestra el Título

\begin{figure}
	\centering
	\includegraphics[scale=0.8]{logo_uni_hamburg.png}
\end{figure}

\newpage %inserta un salto de página




%----------------------------------------------------------------------------------------
%	Question 1
%----------------------------------------------------------------------------------------

\section{Exercise 1.1}

\subsection{A core capability a student of computer science should obtain is to characterise 
and to model an environment for applying an algorithm or in our case a knowledge-based method.
For each distinction give a reason why the difference might be important when designing AI applications for the given environment.
Environments can be characterized by the following concepts:}

	\begin{itemize}
		\item \large{ \textbf{Fully observable <=> partially observable}}
	 	\item \large{ \textbf{Discrete <=> continuous}}
	 	\item \large{ \textbf{Deterministic <=> stochastic}}
	\end{itemize}

\large{ \textbf{Try to identify problems that exists in environments of one type but not in the other. You can use simple examples of environments and tasks to explain these problems.}}
\newline

\begin{itemize}
\item Fully observable: agent sensors give you access to the complete state of the environment; sensors detect all aspects relevant to decision making \\
\item Partially observable: not fully observable due to noise and inaccurate sensors or missing system information \\
\item Discreet: When there is a finite number of states, actions and perceptions.\\
\item Continuous: Does not have a finite number of states.\\
\item Deterministic: if the next state of the environment is totally determined by its current state and the action executed by the agent. \\      
\item Stochastic: not deterministic i. e. the medium is not determined by the current state. \\
\end{itemize}
\newpage



%----------------------------------------------------------------------------------------
%	Question 2
%----------------------------------------------------------------------------------------

\section{Exercise 1.2}


\subsection{Imagine you have to design a route planning application for public transport (German: Öffentlicher Nahverkehr). What would be the state-space in this application?And what would be the semantic of nodes and edges in the graphs representing the state-space for this application?}

\newpage

\subsection{You are given two jugs, a 4-liter jug and a 3-liter jug. Neither has any measuring markers on it. There is a pump that can be used to fill the jugs with water. How can you get exactly 2 liters of water into the 4-liter jug? Solve this riddle using your knowledge about searching gained so far.}

\large{ \textbf{a) Develop a formal model of the problem and select appropriate methods to solveit.}}

\newpage
\large{ \textbf{b)Imagine instead of water you would have to measure 2 liters of expensive wine,so wasting any liquid is not an option. Is the riddle still solvable? Why/Why not?}}

\newpage



%------------------------------------------------

%\bibliography{citas} %archivo citas.bib que contiene las entradas 
%\bibliographystyle{plain} % hay varias formas de citar

\end{document}

