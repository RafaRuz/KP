%%%%%%%%%%%%%%%%%%%%%%%%%%%%%%%%%%%%%%%%%%
% Short Sectioned Assignment LaTeX Template Version 1.0 (5/5/12)
% This template has been downloaded from: http://www.LaTeXTemplates.com
% Original author:  Frits Wenneker (http://www.howtotex.com)
% License: CC BY-NC-SA 3.0 (http://creativecommons.org/licenses/by-nc-sa/3.0/)
%%%%%%%%%%%%%%%%%%%%%%%%%%%%%%%%%%%%%%%%%

%----------------------------------------------------------------------------------------
%	PACKAGES AND OTHER DOCUMENT CONFIGURATIONS
%----------------------------------------------------------------------------------------

\documentclass[paper=a4, fontsize=11pt]{scrartcl} % A4 paper and 11pt font size

% ---- Entrada y salida de texto -----

\usepackage[T1]{fontenc} % Use 8-bit encoding that has 256 glyphs
\usepackage[utf8]{inputenc}
%\usepackage{fourier} % Use the Adobe Utopia font for the document - comment this line to return to the LaTeX default

% ---- Idioma --------

\usepackage[spanish, es-tabla]{babel} % Selecciona el español para palabras introducidas automáticamente, p.ej. "septiembre" en la fecha y especifica que se use la palabra Tabla en vez de Cuadro

% ---- Otros paquetes ----

\usepackage{url} % ,href} %para incluir URLs e hipervínculos dentro del texto (aunque hay que instalar href)
\usepackage{amsmath,amsfonts,amsthm} % Math packages
%\usepackage{graphics,graphicx, floatrow} %para incluir imágenes y notas en las imágenes
\usepackage{graphics,graphicx, float} %para incluir imágenes y colocarlas
\usepackage{verbatim} % comentarios
\usepackage{subfig} %Pa poner figuras paralelas

\usepackage{sidecap}

% Para hacer tablas comlejas

%\usepackage{multirow}
%\usepackage{threeparttable}

%\usepackage{sectsty} % Allows customizing section commands
%\allsectionsfont{\centering \normalfont\scshape} % Make all sections centered, the default font and small caps

\usepackage{fancyhdr} % Custom headers and footers
\pagestyle{fancyplain} % Makes all pages in the document conform to the custom headers and footers
\fancyhead{} % No page header - if you want one, create it in the same way as the footers below
\fancyfoot[L]{} % Empty left footer
\fancyfoot[C]{} % Empty center footer
\fancyfoot[R]{\thepage} % Page numbering for right footer
\renewcommand{\headrulewidth}{0pt} % Remove header underlines
\renewcommand{\footrulewidth}{0pt} % Remove footer underlines
\setlength{\headheight}{13.6pt} % Customize the height of the header

\numberwithin{equation}{section} % Number equations within sections (i.e. 1.1, 1.2, 2.1, 2.2 instead of 1, 2, 3, 4)
\numberwithin{figure}{section} % Number figures within sections (i.e. 1.1, 1.2, 2.1, 2.2 instead of 1, 2, 3, 4)
\numberwithin{table}{section} % Number tables within sections (i.e. 1.1, 1.2, 2.1, 2.2 instead of 1, 2, 3, 4)

\setlength\parindent{0pt} % Removes all indentation from paragraphs - comment this line for an assignment with lots of text

\newcommand{\horrule}[1]{\rule{\linewidth}{#1}} % Create horizontal rule command with 1 argument of height






%----------------------------------------------------------------------------------------
%	TÍTULO Y DATOS DEL ALUMNO
%----------------------------------------------------------------------------------------

\title{	
\normalfont \normalsize 
\textsc{\textbf{Grundlagen der Wissensverarbeitung} \\ Computer Science \\ Universität Hamburg} \\ [25pt] % Your university, school and/or department name(s)
~\\
~\\
~\\
\horrule{0.5pt} \\[0.4cm] % Thin top horizontal rule
\Huge Tutorial 9 : Belief Networks\\ % The assignment title
\horrule{2pt} \\[0.5cm] % Thick bottom horizontal rule
~\\
~\\
}

\author{Rafael Ruz Gómez\\Miguel Robles Urquiza} % Nombre y apellidos

\date{\normalsize \today} % Incluye la fecha actual

%----------------------------------------------------------------------------------------
% DOCUMENTO
%----------------------------------------------------------------------------------------

\begin{document}

\maketitle % Muestra el Título

\begin{figure}
	\centering
	\includegraphics[scale=0.8]{logo_uni_hamburg.png}
\end{figure}

\newpage %inserta un salto de página




%----------------------------------------------------------------------------------------
%	Question 1
%----------------------------------------------------------------------------------------

\section*{Exercise 9.2}

\subsection*{Describe properties of the resulting sequences: what are the similarities and differences to “real” texts?}

Similarities are that sometimes the grammar is correct (excepting the words that has the same form for sustantive and adjective for example, where the program can't distinguish between them or more difficult constructions like weil and the verb at the end) but the whole sentences don't usually makes any sense.


%----------------------------------------------------------------------------------------
%	Question 2
%----------------------------------------------------------------------------------------


\section*{Exercise 9.3}
\subsection*{Compute the following probabilities:}
\begin{itemize}

\item P(Battery\_Wk) = $P(Battery\_NotBrk) = 0.9$\\

\item P(Starter\_Wk) = $P(Starter\_NotBrk)* P(IgnitionKey\_Wk)*P(Battery\_Wk) = 0.9 * P(IgnitionKey\_NotBrk\|Battery\_Wk) * 0.9 =\\ 
0.9 * P(Battery\_Wk||IgnitionKey\_NotBrk) * P (IgnitionKey\_NotBrk) * 0.9 =\\ 0.9*0.9*0.9*0.9 = 0.6561$

\item P(Engine\_Wk) = $P(Engine\_NotBrk) * P(Starter\_Wk) * P(Filter\_Wk) = \\
0.9*0.6561*P(Filter\_NotBrk||(FuelPump\_NotBrk||(FuelTank\_Wk,EFR\_Wk))=
0.9*0.6561*P(Filter\_NotBrk||(FuelPump\_NotBrk||(FuelTank\_NotBrk,\\(EFR\_NotBrk,Battery\_Wk, IgnitionKwy\_Wk))=\\
0.9*0.6561*0.9*0.9*0.9*0.9*0.9*0.9*0.9 = 0,282429536$


\item P(Engine\_Wk|| FuelPump\_Wk) = $P(Engine\_NotBrk) * P(Starter\_Wk) * P(Filter\_Wk) = 
0.9*0.6561*0.9 = 0,531441$

\end{itemize}


%----------------------------------------------------------------------------------------
%	Question 3
%----------------------------------------------------------------------------------------

\newpage

\section*{Exercise 9.4}
\subsection*{Complete the missing probabilities and draw the probability network}


\begin{tabular}{|l|||l ||l |}

\hline
Smuggler &F & 0.99  \\
\hline
 & T & 0.01 \\
\hline
\hline
Fever & F & 0.987  \\
\hline
 & T & 0.013\\
\hline
\end{tabular}
\newline

\begin{tabular}{|l|||l||| l ||l|}
\hline
Fever & Smuggler & Sweat & Probability\\
\hline
F & F & F & 1\\
\hline
F & F & T & 0 \\
\hline
F & T & F & 0.6\\
\hline

F & T & T & 0.4\\
\hline
T & F & F & 0.4\\
\hline
T & F & T & 0.6\\
\hline
T & T & F & 0.2\\
\hline
T & T & T & 0.8\\
\hline

\end{tabular}
\newline

\begin{tabular}{|l|||l||l|}
\hline
Smuggler & DogBk & Probablity\\
\hline
F & F & 0.95\\
\hline
F & T & 0.05\\
\hline
T & F & 0.2\\
\hline
T & T & 0.8\\
\hline
\end{tabular}





\subsection*{Give an example of 'explaining away' in the given network}

If someone is not a smuggler nor has a fever then the likelihood that that person is sweating is zero, so sweating has ben explained away by fever and been smugger.

\subsection*{Compute the following probabilities}
\begin{itemize}

\item P(Smuggler||DogBk) = (DogBk||Smuggler) * P(Smuggler) = 0.8*0.001 = 0.0008

\item P(Sweating) =P(Sw||Fev,Smug) + P(Sw||~Fev,Smug) + P(Sw||Fev,~Smug) + P(Sw||~Fev,~Smug) = 0.99 * 0.987 * 0 + 0.01 * 0.987 * 0.4 + 0.99 * 0.013 * 0.6 + 0.01 * 0.013 * 0.8 = 0.011774


\end{itemize}
%------------------------------------------------

%\bibliography{citas} %archivo citas.bib que contiene las entradas 
%\bibliographystyle{plain} % hay varias formas de citar

\end{document}

