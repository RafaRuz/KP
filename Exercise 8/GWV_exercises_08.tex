%%%%%%%%%%%%%%%%%%%%%%%%%%%%%%%%%%%%%%%%%%
% Short Sectioned Assignment LaTeX Template Version 1.0 (5/5/12)
% This template has been downloaded from: http://www.LaTeXTemplates.com
% Original author:  Frits Wenneker (http://www.howtotex.com)
% License: CC BY-NC-SA 3.0 (http://creativecommons.org/licenses/by-nc-sa/3.0/)
%%%%%%%%%%%%%%%%%%%%%%%%%%%%%%%%%%%%%%%%%

%----------------------------------------------------------------------------------------
%	PACKAGES AND OTHER DOCUMENT CONFIGURATIONS
%----------------------------------------------------------------------------------------

\documentclass[paper=a4, fontsize=11pt]{scrartcl} % A4 paper and 11pt font size

% ---- Entrada y salida de texto -----

\usepackage[T1]{fontenc} % Use 8-bit encoding that has 256 glyphs
\usepackage[utf8]{inputenc}
%\usepackage{fourier} % Use the Adobe Utopia font for the document - comment this line to return to the LaTeX default

% ---- Idioma --------

\usepackage[spanish, es-tabla]{babel} % Selecciona el español para palabras introducidas automáticamente, p.ej. "septiembre" en la fecha y especifica que se use la palabra Tabla en vez de Cuadro

% ---- Otros paquetes ----

\usepackage{url} % ,href} %para incluir URLs e hipervínculos dentro del texto (aunque hay que instalar href)
\usepackage{amsmath,amsfonts,amsthm} % Math packages
%\usepackage{graphics,graphicx, floatrow} %para incluir imágenes y notas en las imágenes
\usepackage{graphics,graphicx, float} %para incluir imágenes y colocarlas
\usepackage{verbatim} % comentarios

% Para hacer tablas comlejas
%\usepackage{multirow}
%\usepackage{threeparttable}

%\usepackage{sectsty} % Allows customizing section commands
%\allsectionsfont{\centering \normalfont\scshape} % Make all sections centered, the default font and small caps

\usepackage{fancyhdr} % Custom headers and footers
\pagestyle{fancyplain} % Makes all pages in the document conform to the custom headers and footers
\fancyhead{} % No page header - if you want one, create it in the same way as the footers below
\fancyfoot[L]{} % Empty left footer
\fancyfoot[C]{} % Empty center footer
\fancyfoot[R]{\thepage} % Page numbering for right footer
\renewcommand{\headrulewidth}{0pt} % Remove header underlines
\renewcommand{\footrulewidth}{0pt} % Remove footer underlines
\setlength{\headheight}{13.6pt} % Customize the height of the header

\numberwithin{equation}{section} % Number equations within sections (i.e. 1.1, 1.2, 2.1, 2.2 instead of 1, 2, 3, 4)
\numberwithin{figure}{section} % Number figures within sections (i.e. 1.1, 1.2, 2.1, 2.2 instead of 1, 2, 3, 4)
\numberwithin{table}{section} % Number tables within sections (i.e. 1.1, 1.2, 2.1, 2.2 instead of 1, 2, 3, 4)

\setlength\parindent{0pt} % Removes all indentation from paragraphs - comment this line for an assignment with lots of text

\newcommand{\horrule}[1]{\rule{\linewidth}{#1}} % Create horizontal rule command with 1 argument of height






%----------------------------------------------------------------------------------------
%	TÍTULO Y DATOS DEL ALUMNO
%----------------------------------------------------------------------------------------

\title{	
\normalfont \normalsize 
\textsc{\textbf{Grundlagen der Wissensverarbeitung} \\ Computer Science \\ Universität Hamburg} \\ [25pt] % Your university, school and/or department name(s)
~\\
~\\
~\\
\horrule{0.5pt} \\[0.4cm] % Thin top horizontal rule
\Huge Tutorial 8 : Propositions and Inference\\ % The assignment title
\horrule{2pt} \\[0.5cm] % Thick bottom horizontal rule
~\\
~\\
}

\author{Rafael Ruz Gómez\\Miguel Robles Urquiza} % Nombre y apellidos

\date{\normalsize \today} % Incluye la fecha actual

%----------------------------------------------------------------------------------------
% DOCUMENTO
%----------------------------------------------------------------------------------------

\begin{document}

\maketitle % Muestra el Título

\begin{figure}
	\centering
	\includegraphics[scale=0.8]{logo_uni_hamburg.png}
\end{figure}

\newpage %inserta un salto de página




%----------------------------------------------------------------------------------------
%	Question 1
%----------------------------------------------------------------------------------------

\section*{Exercise 8.1}

\subsection*{Introduction to Diagnosis: A Murder Investigation
Apply your knowledge of propositions and inference to catch the murderer. Find a formal representation of the assumables, observations, rules and integrity constraints given below. Then compute the minimal conflict and the minimal diagnosis}

Notation:

\begin{itemize}

\item Gardener $\Rightarrow$ G
\item Butler $\Rightarrow$ B
\item Work $\Rightarrow$ w
\item Dirt $\Rightarrow$ d
\item Statement\_Gardener $\Rightarrow$ S\_G
\item Statement\_Butler $\Rightarrow$ S\_B

\end{itemize}


We can see that the rules determine that:

\begin{itemize}
\item S\_G  	= $d_G \leftarrow w_G$ \hspace{4cm} ($d_G$ = 0 $w_G$ = 1)
\item S\_B 	= $d_B \leftarrow w_B$ \hspace{4cm} ($d_B$ = 1 $w_B$ = 1)\\

\end{itemize}
We can conclude that the Gardener's sentence is fake $\Rightarrow$ \textit{Gardener's the murderer}

%----------------------------------------------------------------------------------------
%	Question 2
%----------------------------------------------------------------------------------------

\newpage

\section*{Exercise 8.2}
\subsection*{When the ignition key is turned a good mechanic can hear (observe) three noises produced by the starter, the fuel pump and the engine itself. In case one of the noises is not observed there is a fault in at least one component. Formalize the given diagnosis problem. You can assume that all connections (cables, pipes and mechanical links) work properly but each component (grey box) could be faulty in one way or the other: A fuel tank could be empty, a starter broken, a filter clogged and so on. Perform a diagnosis (that is compute the minimal diagnoses) for the following sets of observations:. }


\begin{itemize}

	\item $ starter(noise1) \leftarrow ignition\_key $
	\item $ ignition\_key \leftarrow battery $
	
	\item $ fuel\_pump(noise2) \leftarrow electronic\_fuel\_regulation \land fuel\_tank $					
	\item $ electronic\_fuel\_regulation \leftarrow battery \land ignition\_key $
			

	\item $ engine(noise3) \leftarrow starter \land filter $
	\item $ filter \leftarrow fuel\_pump $
	
	
	\begin{itemize}		
		\item If there's no noise, then the starter, the fuel pump and the engine are not working. If the starter is not working it's because it's broken or the ignition key is not working. If the ignition key is not working, it's because it's broken or the battery is not working. If we keep going on with this reasoning, we find the following:\\
		
%----------------------------------------------
%                  NO NOISE
%----------------------------------------------		
		$ No\_noise \rightarrow \lnot starter \land \lnot fuel\_pump \land \lnot engine = 1 $\\
		
		And that means: \\
		
		$ (\lnot starter \lor \lnot ignition\_key \lor \lnot battery) \land (\lnot fuel\_pump \lor \lnot fuel\_tank \lor \lnot electronic\_fuel\_regulation \lor \lnot ignition\_key \lor \lnot battery) \land (\lnot engine \lor \lnot filter \lor \lnot fuel\_pump \lor \lnot fuel\_tank \lor \lnot electronic\_fuel\_regulation \lor \lnot ignition\_key \lor \lnot battery \lor \lnot starter \lor \lnot ignition\_key \lor \lnot battery) = 1 $
		
			
		We can not conclude anything
		
\newpage
		
		
%----------------------------------------------
%                  NOISE  1
%----------------------------------------------			
		
		
		\item If we only can hear noise1:
		
		$ Only\_noise1 \rightarrow  starter \land \lnot fuel\_pump \land \lnot engine = 1 $\\
		
		And that means: \\
		
	$ (starter \lor ignition\_key \lor battery) \land (\lnot fuel\_pump \lor \lnot fuel\_tank \lor \lnot electronic\_fuel\_regulation \lor \lnot ignition\_key \lor \lnot battery) \land (\lnot engine \lor \lnot filter \lor \lnot fuel\_pump \lor \lnot fuel\_tank \lor \lnot electronic\_fuel\_regulation \lor \lnot ignition\_key \lor \lnot battery \lor \lnot starter \lor \lnot ignition\_key \lor \lnot battery) = 1 $
		
		
		We conclude that the battery, the ignition key and the starter are working, but the electronic fuel regulation or fuel pump are broken at least.
		
		
%----------------------------------------------
%                  NOISE  2
%----------------------------------------------			
		
		
\newpage		
		\item If we only can hear noise2:
		
		$ Only\_noise1 \rightarrow  \lnot starter \land fuel\_pump \land \lnot engine = 1 $\\
		
		And that means: \\
		
$ (\lnot starter \lor \lnot ignition\_key \lor \lnot battery) \land ( fuel\_pump \lor fuel\_tank \lor electronic\_fuel\_regulation \lor ignition\_key \lor battery) \land (\lnot engine \lor \lnot filter \lor \lnot fuel\_pump \lor \lnot fuel\_tank \lor \lnot electronic\_fuel\_regulation \lor \lnot ignition\_key \lor \lnot battery \lor \lnot starter \lor \lnot ignition\_key \lor \lnot battery) = 1 $
		
		
		We conclude that the battery, the ignition key, the electronic fuel regulation, the fuel pump and the fuel tank are working, but the filter or the engine are broken at least and the stater is broken.
		
%----------------------------------------------
%                  NOT NOISE  3
%----------------------------------------------			
		
\newpage		
		\item If we can hear noise1 and noise 2 but not noise3:
		
		$ Only\_noise1 \rightarrow  starter \land fuel\_pump \land \lnot engine = 1 $\\
		
		And that means: \\
		
$ ( starter \lor  ignition\_key \lor  battery) \land ( fuel\_pump \lor fuel\_tank \lor electronic\_fuel\_regulation \lor ignition\_key \lor battery) \land (\lnot engine \lor \lnot filter \lor \lnot fuel\_pump \lor \lnot fuel\_tank \lor \lnot electronic\_fuel\_regulation \lor \lnot ignition\_key \lor \lnot battery \lor \lnot starter \lor \lnot ignition\_key \lor \lnot battery) = 1 $
		
		
		We conclude that the battery, the ignition key, the starter, the electronic fuel regulation, the fuel pump and the fuel tank are working, but the filter or the engine are broken at least.
		
	\end{itemize}

	\begin{comment}

		\begin{itemize}
			\item No noise\\
				If no noise1, then battery is broken or ignition key is broken\\
				If no noise2, then battery is broken or ignition key is broken or fuel tank is broken\\
				If no noise3, then there is at least some element broken\\
				
			\item Only noise1\\
				If only noise1, then battery works and works\\
				If no noise2 and electronic fuel regulation works, then fuel tank is broken\\
				
			\item Only noise2\\	
				If only noise 2, then electronic fuel regulation works and fuel tank works
				If no noise1 and electronic fuel regulation works, then batery and ingnition key works but starter is broken
			\item Noise1 and 2 but not 3\\
				If noise1 then ingnition key and battery works
				If noise 2 then electronic fuel regulation and fuel tank works 
				If no noise 3 and starter and fuel pump works, then engine or filter are broken
			
			
		\end{itemize}
		
\end{comment}

\end{itemize}

If a variable is false, it means it's not working and so on it could be broken.



%------------------------------------------------

%\bibliography{citas} %archivo citas.bib que contiene las entradas 
%\bibliographystyle{plain} % hay varias formas de citar

\end{document}

