%%%%%%%%%%%%%%%%%%%%%%%%%%%%%%%%%%%%%%%%%%
% Short Sectioned Assignment LaTeX Template Version 1.0 (5/5/12)
% This template has been downloaded from: http://www.LaTeXTemplates.com
% Original author:  Frits Wenneker (http://www.howtotex.com)
% License: CC BY-NC-SA 3.0 (http://creativecommons.org/licenses/by-nc-sa/3.0/)
%%%%%%%%%%%%%%%%%%%%%%%%%%%%%%%%%%%%%%%%%

%----------------------------------------------------------------------------------------
%	PACKAGES AND OTHER DOCUMENT CONFIGURATIONS
%----------------------------------------------------------------------------------------

\documentclass[paper=a4, fontsize=11pt]{scrartcl} % A4 paper and 11pt font size

% ---- Entrada y salida de texto -----

\usepackage[T1]{fontenc} % Use 8-bit encoding that has 256 glyphs
\usepackage[utf8]{inputenc}
%\usepackage{fourier} % Use the Adobe Utopia font for the document - comment this line to return to the LaTeX default

% ---- Idioma --------

\usepackage[spanish, es-tabla]{babel} % Selecciona el español para palabras introducidas automáticamente, p.ej. "septiembre" en la fecha y especifica que se use la palabra Tabla en vez de Cuadro

% ---- Otros paquetes ----

\usepackage{url} % ,href} %para incluir URLs e hipervínculos dentro del texto (aunque hay que instalar href)
\usepackage{amsmath,amsfonts,amsthm} % Math packages
%\usepackage{graphics,graphicx, floatrow} %para incluir imágenes y notas en las imágenes
\usepackage{graphics,graphicx, float} %para incluir imágenes y colocarlas

% Para hacer tablas comlejas
%\usepackage{multirow}
%\usepackage{threeparttable}

%\usepackage{sectsty} % Allows customizing section commands
%\allsectionsfont{\centering \normalfont\scshape} % Make all sections centered, the default font and small caps

\usepackage{fancyhdr} % Custom headers and footers
\pagestyle{fancyplain} % Makes all pages in the document conform to the custom headers and footers
\fancyhead{} % No page header - if you want one, create it in the same way as the footers below
\fancyfoot[L]{} % Empty left footer
\fancyfoot[C]{} % Empty center footer
\fancyfoot[R]{\thepage} % Page numbering for right footer
\renewcommand{\headrulewidth}{0pt} % Remove header underlines
\renewcommand{\footrulewidth}{0pt} % Remove footer underlines
\setlength{\headheight}{13.6pt} % Customize the height of the header

\numberwithin{equation}{section} % Number equations within sections (i.e. 1.1, 1.2, 2.1, 2.2 instead of 1, 2, 3, 4)
\numberwithin{figure}{section} % Number figures within sections (i.e. 1.1, 1.2, 2.1, 2.2 instead of 1, 2, 3, 4)
\numberwithin{table}{section} % Number tables within sections (i.e. 1.1, 1.2, 2.1, 2.2 instead of 1, 2, 3, 4)

\setlength\parindent{0pt} % Removes all indentation from paragraphs - comment this line for an assignment with lots of text

\newcommand{\horrule}[1]{\rule{\linewidth}{#1}} % Create horizontal rule command with 1 argument of height






%----------------------------------------------------------------------------------------
%	TÍTULO Y DATOS DEL ALUMNO
%----------------------------------------------------------------------------------------

\title{	
\normalfont \normalsize 
\textsc{\textbf{Grundlagen der Wissensverarbeitung} \\ Computer Science \\ Universität Hamburg} \\ [25pt] % Your university, school and/or department name(s)
~\\
~\\
~\\
\horrule{0.5pt} \\[0.4cm] % Thin top horizontal rule
\Huge Tutorial 7: Constraint Satisfaction\\ % The assignment title
\horrule{2pt} \\[0.5cm] % Thick bottom horizontal rule
~\\
~\\
}

\author{Rafael Ruz Gómez\\Miguel Robles Urquiza} % Nombre y apellidos

\date{\normalsize \today} % Incluye la fecha actual

%----------------------------------------------------------------------------------------
% DOCUMENTO
%----------------------------------------------------------------------------------------

\begin{document}

\maketitle % Muestra el Título

\begin{figure}
	\centering
	\includegraphics[scale=0.8]{logo_uni_hamburg.png}
\end{figure}

\newpage %inserta un salto de página




%----------------------------------------------------------------------------------------
%	Question 1
%----------------------------------------------------------------------------------------

\section*{Exercise 7.1}

\subsection*{Formalize this riddle in the form of a constraint network, with the constraints being reasonably small. (I.e. writing a single constraint is not a good solution!) In the following pattern each letter stands for a digit so that the resulting sum is correct.}
\textbf{
\ \ \ \ S E N D\\
+\  M O R E\\
=======\\
\ M O N E Y\\
}

Let's call $S_i$ the sum done in the last step. Then the constraints are:
\begin{itemize}
	\item Different letter have different values: $S \neq E , E \neq N , N \neq D...$
	\item $(D + E) \% 10 == Y \ \ \ \ \ \ ( S_0 = (D + E) )$
	\item $ (((S_0 - S_0) \% 10) / 10 + N + R) \% 10 == E \ \ \ \ \ \ ( S_1 = ((S_0 - S_0) \% 10) / 10 + N + R ) $
	\item $ (((S_1 - S_1) \% 10) / 10 + E + O) \% 10 == N \ \ \ \ \ \ ( S_2 = ((S_1 - S_1) \% 10) / 10 + E + O) $
	\item $ (((S_2 - S_2) \% 10) / 10 + S + M) \% 10 == O \ \ \ \ \ \ ( S_3 = ((S_2 - S_2) \% 10) / 10 + S + M) $
	\item $ (((S_3 - S_3) \% 10) / 10 == M $
\end{itemize}
%----------------------------------------------------------------------------------------
%	Question 2
%----------------------------------------------------------------------------------------

\subsection*{Manual constraint solving.}

\textbf{
Crossword puzzles are often used in newspapers because they provide joy in solving semi-complex problems by combining logics and human experience. For the crossword above we want to find 6 words of length 3 that fit into the 3x3 table in a way that 3 words can be read horizontal from left to right and 3 words can be read vertically from top to bottom. Choose the words from the following list:\\
\\
add, ado, age, ago, aid, ail, aim, air,\\
and, any, ape, apt, arc, are, ark, arm,\\
art, ash, ask, auk, awe, awl, aye, bad,\\
bag, ban, bat, bee, boa, ear, eel, eft,\\
far, fat, fit, lee, oaf, rat, tar, tie.\\
\\
First, try to solve the problem without any formal methods or tools. How do you approach this problem as a human? (It is not necessary to give a full solution to the problem at this point, but you should report on the strategies you employ as a human and the problems you encounter.)
}



%----------------------------------------------------------------------------------------
%	Question 3
%----------------------------------------------------------------------------------------

\subsection*{Solve the problem by hand using domain consistency as a first step and as a second step the arc consistency. Document this process thoroughly.}

%----------------------------------------------------------------------------------------
%	Question 4
%----------------------------------------------------------------------------------------


\subsection*{• Implement the arc consistency algorithm (found in sect. 4.5 of Poole and Mackworth (2010)) along with a suitable representation of the problem to solve this puzzle.}






%------------------------------------------------

%\bibliography{citas} %archivo citas.bib que contiene las entradas 
%\bibliographystyle{plain} % hay varias formas de citar

\end{document}

